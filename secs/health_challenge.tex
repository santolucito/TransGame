\section{Please describe the health challenge your organization or product is attempting to address}
%(100 words maximum)

These grants are for \textit{"disproportionately impacts low-income communities in the United States or low-resource countries."}

% 1. Access to psychological care is limited by the number of practitioners exist in the U.S - as of June 2014, the APA (American Psychological Association) estimated 106,500 psychologists possess current licenses in the US \cite{howmanypsychs}.

As mental health continues to develop as a core component of societal wellness, we need to rethink the way in which we address these issues. For many problems of this type, there is no simple drug or treatment than can be administered. Instead, care is often delivered by psychologists or other mental health professionals, often in the form of structured therapies. The precise form of the therapy may depend on the particular issue being addressed (a psychological disorder, emotional trauma, etc) but the format of treatment fundamentally requires the presence and expertise of a trained expert.

\crahul{It is for precisely this reason that our current structures suffer from a problem of scale. As we begin to think about mental health as a continuous approach towards wellness rather than a set of discrete illnesses to be treated, we will simply not have enough people to use our current methods to reach out to all the people we need to.
}

Transgender people are often considered outsiders not only by society at large, but also by their families when they decide to come out. The dissolution of families is a major root cause of emotional trauma for transpeople\cite{simons2013parental}, and loss of financial backing can often force them into dangerous professions such as sex work. As such, we hope to improve the transition process for millions of transgender people by building tools that help to ensure that critical family support structures remain intact.

% would probably be nice to have some general demographic stats
% here - like X% of transpeople with family support have Y%
% better transition over some {SET} of definable metrics.

In addition, this series of games, will address a demographic that is currently completely underserved - the families of transpeople. This demographic is given little to no support during the transition process. Addressing the mental health of the support structures for transpeople will allow everyone to avoid expensive (emotionally, physically, financially) misunderstandings and conflict.




