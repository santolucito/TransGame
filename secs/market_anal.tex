

\section{Market Analysis} 
%(500 words maximum)
 
%Please describe any completed market research
\begin{enumerate}
\itemsep-0.5em 
    \item How many potential beneficiaries/customers have you spoken to? (Of what types, What did you learn, etc.)
    \item How do you know people need what you’re making/providing?
\end{enumerate}

The project lead, Mark Santolucito, has recently experienced a transition in his family. While the transition was relatively smooth, the other 1.4 million transgender people in the U.S. are not always as lucky.
We have attended 

We conducted two in-depth follow up interviews after beta testing of our first game.
While we had assumed one of the main difficulties in a transition was the switching of pronouns, in fact communication was the biggest pain point.

We say this problem echoed in transgender support groups.
We have attended two support group meeting for families of transgender people.
One at Cooley-Dickinson  Hospital  in  Northampton,  MA,  and  one  at  a  PFLAG  meeting  in Manchester,  CT. 
PFLAG  is  a  national  organization  for  parents,families,  and friends of LGBTQ people. 
After these experiences we have pivoted slightly to focus on a lack of communication as a root cause of mental distress.

In a more informal sense, Mark also was in Thailand to support his sister, Fiona, during her Sexual Reassignment Surgery (SRS).
The vast majority of patients were alone and without family.
In one conversation with a Brazilian girl, she said that he family had completely stopped communicating with her - not an uncommon experience.
SRS and the recovery is an incredibly physically and emotionally taxing experience.
In post-op, Fiona said she could not imagine trying to do it without the support of her family.

%Describe your competitors
\begin{enumerate}
\itemsep-0.5em 
    \item Are there other groups that provide identical or similar products/services?
    \item Highlight your potential relevant competitors and describe your competitive advantage
\end{enumerate}

\textbf{Project Evo} has raised \$42 million in venture funding and is now seeking FDA approval of a game as ADHD treatment. They are also developing tools for Autism, depression, Alzheimer’s disease, and traumatic brain injury~\cite{akili}.

While these games are broadening the standard of care for important psychiatric conditions, they are primarily focused on {\it games played directly by patients}. We intend to further this already proven model by expanding it to include a wider social environment. As mental health is critically affected by the social interactions between people and their environment, we believe that it is not possible to ignore this effect when attempting to treat mental health problems, especially those which are socially stigmatized.

As mental health is critically affected by the social interactions between people (I have a REF http://www.latimes.com/science/sciencenow/la-sci-sn-brain-training-dementia-20160724-snap-story.html at the end of the article Wisconsin Alzheimer’s Institute) and their environment, we believe that it is not possible to ignore this effect when attempting to treat mental health problems, especially those which are socially stigmatized.

However, they are not a true competitor since they are not targeting social issues. The market for games as psychiatric treatment remains open, but we compete with pharmaceuticals and in-person therapy. We are proposing an alternative (or supplement) to such traditional treatments.

HopeLabs \url{http://www.hopelab.org/about/} is a nonprofit using technology for health. They build a game to help kids understand cancer treatment, as well as an autotexter that send nice messages to kids that encouraged them to be better people.

The Advanced Cognitive Training for Independent and Vital Elderly (ACTIVE) study was funded by the National Institute on Aging and its results have just been published at Alzheimer's Association International Conference (2016). The ACTIVE study used brain-training exercise-based games such as Double Decision relying on player’s ability to detect, remember, and respond to cues that (dis)appear quickly in varying locations on a computer screen. Players in the high speed-of-processing gameplay group showed less cognitive decline over the 10-year study. +