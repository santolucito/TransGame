\section{Full summary of the new venture}
%(250 words maximum)

%- *What* is the venture?

\app delivers clinical-grade psychotherapy at scale to transgender people and their families - using proven psychological intervention methods through everyday devices to combat mental illness. 

%- *What* is the innovation / novel value proposition?

By using technology as a distribution method, \app is able to reach more people at lower cost than any mainstream method of delivering psychological intervention.

%- *How large* is the opportunity?


%- *What* will define success?

We will define the application as having significant impact if 

%- *When* will the product be built? (maybe)

We plan to roll out the product over the course of the next 9 months - We will build the first application over the course of 3 months, run a study to validate clinical significance in the following 3 months, and distribute the application to doctors and support groups in the final 3 months.

%- *How* will it be distributed?

The venture will target two major channels of distribution: family support groups and psychologists (every trans person on HRT has a psychologist)

%- *Why* is this the right team for this venture?


%- *How* will this help specifically low-income communities in the US or low resource countries?


\app delivers clinical-grade psychotherapy at scale to transgender people and their families - using proven psychological intervention methods through everyday devices to combat mental illness. 

By using technology as a distribution method, \app is able to reach more people at lower cost than any mainstream method of delivering psychological intervention.

The venture will target two major channels of distribution: family support groups and psychologists (every trans person on HRT has a psychologist)

\iffalse
We are building a medical app for the trans community. The app is a suite of games that help give new perspective and empathy to different parties involved in transitions. We plan to target transgender people, their families, and their healthcare providers.

A vast majority of transgender people already identify as "gamers" (probably, have to find some evidence, but wanting to escape reality and all that), and we know that games can be used to have an impact on people's opinions and behaviors~\cite{boyle2011role,hwang2016}. We propose to leverage these two facts to make games specifically targeting the mental health of transgender people.


We distribute through practitioners, then eventually get Food and Drug Administration (FDA) approval so insurance covers costs to the patients.

%theoretical basis
According to Webb and Sheeran \cite{webb2009planning} 
"following the task switch, participants who had developed relatively habitual responses during the training phase found it very difficult to unlearn these responses and task performance was significantly impaired"

There are rising awareness of the importance in communication among family members of the mentally ill or disadvantaged
\cite{russell2016knowledge}.
\fi