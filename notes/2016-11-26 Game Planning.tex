# Space Swiper is diagnostic

The first game (Space Swiper) we have seems it will be better suited as a diagnosis game, rather than interventional.
It can be used to to identify "Habituation strength", which is can be used as an indicator of the success of "if-then behavior change" efforts.


# New game as intervention

Let's make a second game that models the feeling of not fitting into your gender role.

A game where the instructions are implicit, or by example.
There are two teams (genders), each score points with a different set of rules (societal norms).
The player plays by the rules of their team, and sees their points go up.
After the habituation period, the player's scoring is based on the rules of the opposing team, but they are not told when the switch happens.
If they continue to play by the rules of their team, they lose points.
If they switch to play by the rules of the opposing team, they can score points, but still get tactile feedback based on rules of their team.

The rules can also switch back and forth, creating more confusion and frustration for the player.
This game is then a talking point for the family about what the transperson is going through. 

TODO
- a setting for the game
- making the game
